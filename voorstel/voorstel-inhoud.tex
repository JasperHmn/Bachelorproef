%---------- Inleiding ---------------------------------------------------------

% TODO: Is dit voorstel gebaseerd op een paper van Research Methods die je
% vorig jaar hebt ingediend? Heb je daarbij eventueel samengewerkt met een
% andere student?
% Zo ja, haal dan de tekst hieronder uit commentaar en pas aan.

%\paragraph{Opmerking}

% Dit voorstel is gebaseerd op het onderzoeksvoorstel dat werd geschreven in het
% kader van het vak Research Methods dat ik (vorig/dit) academiejaar heb
% uitgewerkt (met medesturent VOORNAAM NAAM als mede-auteur).
% 

\section{Inleiding}%
\label{sec:inleiding}
De opkomst van technologie heeft een grote impact op alle aspecten van het leven, waaronder de 
toegankelijkheid voor mensen met kleurenblindheid. Kleurenblindheid, die verschillende vormen kan 
aannemen zoals protanopie, deuteranopie en tritanopie, ontstaat door defecten in de fotoreceptoren 
in het netvlies, wat resulteert in een verminderd vermogen om kleuren nauwkeurig waar te nemen 
\textcite{Salih2020}. Dit heeft aanzienlijke gevolgen voor de gebruikerservaring, vooral in digitale 
omgevingen zoals mobiele applicaties. De technologische vooruitgang biedt echter mogelijkheden om de 
toegankelijkheid en bruikbaarheid voor individuen met kleurenblindheid te verbeteren. Het ontwikkelen 
van apps en tools die visuele beperkingen compenseren, kan leiden tot een meer inclusieve 
gebruikerservaring \textcite{Bandyopadhyay2017}. Deze studie richt zich op de ontwikkeling en toepassing van een Software 
Development Kit (SDK) die specifiek ontworpen is om de toegankelijkheid van mobiele applicaties 
voor mensen met kleurenblindheid te verbeteren. De SDK omvat functies zoals kleurfilters, 
contrastversterking en geoptimaliseerde kleurenschema’s en zal worden getest binnen een 
softwarebedrijf dat Android-applicaties ontwikkelt. De nadruk ligt op het integreren van deze 
technologieën in bestaande applicatiestructuren zonder dat ingrijpende veranderingen aan de 
architectuur nodig zijn. Hoewel er diverse methoden bestaan, zoals kleurherkenning apps en 
contrastverhogende software, blijken deze oplossingen vaak beperkt in hun toepasbaarheid en 
flexibiliteit \textcite{Baswaraju2020}. Veel van deze apps en tools richten zich op specifieke situaties, zoals het herkennen 
van kleuren in afbeeldingen of het verhogen van het contrast van tekst, maar missen de veelzijdigheid 
om consistent geïntegreerd te worden in bredere digitale omgevingen. Hierdoor blijven veel digitale 
platformen en applicaties ontoereikend in het ondersteunen van gebruikers met kleurenblindheid, met 
als gevolg dat deze gebruikers tegen obstakels aanlopen in hun dagelijkse interacties met 
technologie. Hieruit vloeit de vraag: in hoeverre kan de integratie van een toegespitste SDK, met 
functies zoals kleurfilters en algoritmen die de kleurwaarneming verbeteren, bijdragen aan de 
toegankelijkheid en gebruikerservaring van Android-apps voor mensen met kleurenblindheid?
Om deze vraag te beantwoorden, worden vier deelvragen geformuleerd:

\begin{itemize}
  \item Ten eerste, welke technologieën en algoritmen kunnen het meest effectief worden 
  geïntegreerd in een SDK om de kleurwaarneming in Android-apps voor kleurenblinde gebruikers 
  te verbeteren?
  \item Ten tweede, hoe verhouden de gebruikerservaringen van kleurenblinde gebruikers zich tot 
  die van niet-kleurenblinde gebruikers bij het gebruik van Android-apps die de 
  kleurwaarnemingsverbeteringen van de SDK integreren?
  \item Ten derde, in hoeverre voldoet de ontwikkelde SDK aan de relevante 
  WCAG-richtlijnen voor visuele toegankelijkheid, en welke verbeteringen kunnen nog worden 
  aangebracht?
  \item Ten slotte, welke technische en praktische uitdagingen komen ontwikkelaars tegen bij het 
  integreren van de SDK in bestaande Android-apps, en hoe kan de implementatie worden vereenvoudigd?
\end{itemize}
\bigskip
Gedurende veertien weken worden de apps van een softwarebedrijf aangepast met de ontwikkelde SDK. 
De effectiviteit van deze aanpassingen wordt beoordeeld door middel van gebruikerstests met 5 
kleurenblinde en 5 niet-kleurenblinde testers. De prestaties worden geëvalueerd op basis van 
taakprestaties en gebruiksvriendelijkheid. Daarnaast wordt de naleving van WCAG-richtlijnen 
gecontroleerd met de Google Accessibility Scanner en handmatige beoordelingen. In wat volgt, 
wordt eerst een overzicht gegeven van de huidige stand van zaken binnen het onderzoeksdomein, 
gebaseerd op een uitgebreide literatuurstudie over toegankelijkheid technologieën voor kleurenblinde 
gebruikers voor Android-apps. Vervolgens wordt de methodologie van het onderzoek in detail 
beschreven. Daarna worden de resultaten getoond van de integratie van de algoritmen en filters in 
een SDK en worden deze geanalyseerd op basis van hun effectiviteit en gebruikerservaring. Ten slotte 
wordt een algemene conclusie getrokken, waarin de belangrijkste bevindingen en aanbevelingen worden 
samengevat.
%---------- Stand van zaken ---------------------------------------------------
\section{Literatuurstudie}%
\label{sec:literatuurstudie}
% Voor literatuurverwijzingen zijn er twee belangrijke commando's:
% \autocite{KEY} => (Auteur, jaartal) Gebruik dit als de naam van de auteur
%   geen onderdeel is van de zin.
% \textcite{KEY} => Auteur (jaartal)  Gebruik dit als de auteursnaam wel een
%   functie heeft in de zin (bv. ``Uit onderzoek door Doll & Hill (1954) bleek
%   ...'')

\subsection{Kleurenblindheid op Android}
Kleurenblindheid, waaronder protanopie, deuteranopie en tritanopie, ontstaat door defecten in de 
fotoreceptoren in het netvlies, waardoor het moeilijk wordt om kleuren correct waar te nemen \autocite{Salih2020}. 
Dit heeft invloed op de gebruikerservaring van mensen met deze aandoeningen, vooral in digitale omgevingen. 
Android biedt al verschillende apps die gebruikers helpen bij het herkennen van kleuren, bijvoorbeeld 
in wetenschappelijke toepassingen zoals titraties met fenolftaleïne. Volgens \textcite{Bandyopadhyay2017} verbeteren 
de apps niet alleen de precisie in laboratoriumresultaten, maar vergroten ze ook de algehele gebruikerservaring 
door visuele beperkingen te beperken. Daarnaast zijn er algoritmes zoals LMS Daltonization en LAB-correctie 
ontwikkeld, die de waarneming van kleuren verbeteren, wat vooral nuttig is in digitale en interactieve 
omgevingen \autocite{Baswaraju2020}. Dit toont aan hoe technologische innovaties de toegankelijkheid kunnen vergroten. 
Volgens \textcite{Crawford2024} is het naleven van de WCAG 2.1-normen essentieel voor het waarborgen van de 
toegankelijkheid van apps, zeker met de verplichtingen die zijn vastgelegd in de Europese Toegankelijkheidswet.
De implementatie van deze richtlijnen helpt ontwikkelaars bij het ontwerpen van apps die breder toegankelijk
zijn voor mensen met visuele beperkingen.
\subsection{Android SDK}
De Android Software Development Kit (SDK) biedt ontwikkelaars de nodige tools voor het bouwen 
van apps die niet alleen interactief en responsief zijn, maar ook rekening houden met 
de toegankelijkheid van gebruikers. Dit is van groot belang voor mensen met visuele beperkingen 
zoals kleurenblindheid. De SDK bevat verschillende componenten die essentieel zijn voor het 
ontwikkelen van apps die visueel aangepast kunnen worden aan de behoeften van de gebruiker, 
zoals UI-elementen, content providers en API-integraties. Volgens \textcite{Geeks2024} kunnen ontwikkelaars 
met de Android SDK gebruik maken van functies zoals TalkBack, een schermlezer die tekst hardop 
voorleest en het gemakkelijker maakt om door een app te navigeren voor visueel gehandicapte gebruikers. 
Dit vergroot de bruikbaarheid van apps, vooral voor mensen met kleurenblindheid. De juiste integratie 
van API's en interactieve elementen zijn essentieel voor het creëren van apps die toegankelijk zijn 
voor mensen met verschillende beperkingen. Door deze integratie kan de app zich aanpassen aan de 
visuele behoeften van de gebruiker, wat bijdraagt aan een inclusievere gebruikerservaring \autocite{Lee2011}.
\subsection{Android Toegankelijkheid}
Android biedt een breed scala aan functies die specifiek gericht zijn op het verbeteren van 
de toegankelijkheid van apps. Het is van groot belang dat ontwikkelaars deze functies vanaf de 
ontwerpfase meenemen, zodat de app geschikt is voor gebruikers met visuele beperkingen, zoals 
kleurenblindheid. Functies zoals inhoud beschrijvingen voor afbeeldingen, spraakcommando’s en de 
TalkBack-schermlezer dragen bij aan de bruikbaarheid van apps voor visueel gehandicapte 
gebruikers \autocite{Swearngin2024}. Door deze technologieën te integreren, kunnen ontwikkelaars de 
toegankelijkheid van apps aanzienlijk verbeteren en ervoor zorgen dat deze goed bruikbaar zijn voor 
mensen met verschillende vormen van kleurenblindheid. Het testen van de toegankelijkheid van apps 
tijdens de ontwerpfase is cruciaal, zoals recent onderzoek benadrukt \textcite{Gregorio2022}. Dit stelt 
ontwikkelaars in staat om de interactie van gebruikers met de app te optimaliseren en te zorgen voor 
een naadloze ervaring, ongeacht visuele beperkingen.
\subsection{WCAG Richtlijnen}
De Web Content Accessibility Guidelines (WCAG) vormen een cruciaal kader voor het ontwerpen 
van toegankelijke digitale inhoud. Deze richtlijnen zijn niet alleen van toepassing op websites, 
maar ook op mobiele apps. Het naleven van WCAG-normen is essentieel om apps te ontwerpen die 
toegankelijk zijn voor gebruikers met kleurenblindheid. Een belangrijk aspect van de richtlijnen is 
het kleurcontrast, dat ervoor zorgt dat de inhoud van apps goed leesbaar is, zelfs voor mensen die 
moeite hebben met het onderscheiden van bepaalde kleuren \autocite{Lindahl2023}. Het volgen van deze 
richtlijnen maakt het voor ontwikkelaars mogelijk om apps te creëren die voor een breed scala aan 
gebruikers toegankelijk zijn, wat bijdraagt aan een inclusieve digitale omgeving. Door bewust 
rekening te houden met deze richtlijnen kunnen ontwikkelaars een bredere gebruikersgroep bereiken, 
inclusief mensen met visuele beperkingen.



%---------- Methodologie ------------------------------------------------------
\section{Methodologie}%
\label{sec:methodologie}
\subsection{Fase 1: Literatuurstudie}
In de eerste fase, gedurende week één en twee, wordt een beknopte literatuurstudie uitgevoerd om de 
belangrijkste technologieën en algoritmen te identificeren die relevant zijn voor de verbetering 
van kleurwaarneming. Tegelijkertijd worden gesprekken gevoerd met Android-ontwikkelaars om de 
vereisten voor de SDK te verduidelijken. De resultaten van deze gesprekken worden gedetailleerd 
vastgelegd en dienen als basis voor de volgende fasen van het project.
\subsection{Fase 2: Ontwerp en Ontwikkeling}
De tweede fase, van week drie tot en met acht, omvat het ontwerpen en ontwikkelen van de SDK. 
De nadruk ligt op de implementatie van functies zoals kleurenfilters en contrastversterking. 
Het ontwerp van de SDK wordt modulair en schaalbaar opgezet, zodat integratie in Android-apps 
eenvoudig is. De ontwikkelde code wordt in een gecontroleerde omgeving getest om de stabiliteit 
te waarborgen.
\subsection{Fase 3: Proof of Concept}
In week negen tot en met twaalf wordt de SDK geïntegreerd in verschillende Android-apps 
binnen het bedrijf. Deze fase omvat intensieve tests om de compatibiliteit en stabiliteit van 
de SDK te garanderen. Problemen die tijdens de implementatie naar voren komen, worden aangepakt 
en opgelost om een succesvolle integratie te waarborgen.
\subsection{Fase 4: Evaluatie}
De laatste fase, in week dertien en veertien, is gewijd aan gebruikerstesten. 
De SDK wordt getest door 5 kleurenblinde en 5 niet-kleurenblinde gebruikers om taakprestaties 
en toegankelijkheid te evalueren. De naleving van WCAG-richtlijnen wordt gecontroleerd met behulp 
van de Google Accessibility Scanner en handmatige evaluaties.
%---------- Verwachte resultaten ----------------------------------------------
\section{Verwacht resultaat, conclusie}%
\label{sec:verwachte_resultaten}
Op basis van het literatuuronderzoek en de gehanteerde methodologie wordt verwacht dat de 
integratie van een SDK met functies zoals kleurfilters en verbeterde algoritmen een positieve 
invloed zal hebben op de toegankelijkheid en gebruikerservaring van Android-apps voor kleurenblinde 
gebruikers. De implementatie van de technologieën zal naar verwachting de visuele herkenning 
verbeteren, waardoor gebruikers met kleurenblindheid gemakkelijker kleuren kunnen onderscheiden 
en navigeren binnen de applicatie. Hierdoor zullen hun ervaringen meer aansluiten bij die van 
niet-kleurenblinde gebruikers.
Daarnaast wordt voorzien dat de ontwikkelde SDK grotendeels zal voldoen aan de WCAG-richtlijnen voor 
visuele toegankelijkheid. Dit zal de naleving van deze richtlijnen binnen apps versterken, met 
eventuele kleine aanpassingen om volledige conformiteit te bereiken. Dit aspect zal bijdragen aan 
het verhogen van de algehele toegankelijkheid van de apps en gebruikers meer vertrouwen geven in 
hun gebruik.
De integratie van de SDK zal waarschijnlijk enkele technische en praktische uitdagingen met zich 
meebrengen, zoals de compatibiliteit met bestaande apps en mogelijke prestatieproblemen. Verwacht 
wordt echter dat duidelijke documentatie en modulaire opties de integratie zullen vereenvoudigen 
en obstakels voor ontwikkelaars verminderen.
Ten slotte zal het gebruik van deze SDK bijdragen aan het verkleinen van de kloof in 
gebruikerservaring tussen kleurenblinde en niet-kleurenblinde gebruikers, wat zal bijdragen aan een 
inclusievere digitale omgeving.


