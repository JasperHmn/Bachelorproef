%---------- Inleiding ---------------------------------------------------------

% TODO: Is dit voorstel gebaseerd op een paper van Research Methods die je
% vorig jaar hebt ingediend? Heb je daarbij eventueel samengewerkt met een
% andere student?
% Zo ja, haal dan de tekst hieronder uit commentaar en pas aan.

%\paragraph{Opmerking}

% Dit voorstel is gebaseerd op het onderzoeksvoorstel dat werd geschreven in het
% kader van het vak Research Methods dat ik (vorig/dit) academiejaar heb
% uitgewerkt (met medesturent VOORNAAM NAAM als mede-auteur).
% 

\section{Inleiding}%
\label{sec:inleiding}

Waarover zal je bachelorproef gaan? Introduceer het thema en zorg dat volgende zaken zeker duidelijk aanwezig zijn:

\begin{itemize}
  \item kaderen thema
  \item de doelgroep
  \item de probleemstelling en (centrale) onderzoeksvraag
  \item de onderzoeksdoelstelling
\end{itemize}

Denk er aan: een typische bachelorproef is \textit{toegepast onderzoek}, wat betekent dat je start vanuit een concrete probleemsituatie in bedrijfscontext, een \textbf{casus}. Het is belangrijk om je onderwerp goed af te bakenen: je gaat voor die \textit{ene specifieke probleemsituatie} op zoek naar een goede oplossing, op basis van de huidige kennis in het vakgebied.

De doelgroep moet ook concreet en duidelijk zijn, dus geen algemene of vaag gedefinieerde groepen zoals \emph{bedrijven}, \emph{developers}, \emph{Vlamingen}, enz. Je richt je in elk geval op it-professionals, een bachelorproef is geen populariserende tekst. Eén specifiek bedrijf (die te maken hebben met een concrete probleemsituatie) is dus beter dan \emph{bedrijven} in het algemeen.

Formuleer duidelijk de onderzoeksvraag! De begeleiders lezen nog steeds te veel voorstellen waarin we geen onderzoeksvraag terugvinden.

Schrijf ook iets over de doelstelling. Wat zie je als het concrete eindresultaat van je onderzoek, naast de uitgeschreven scriptie? Is het een proof-of-concept, een rapport met aanbevelingen, \ldots Met welk eindresultaat kan je je bachelorproef als een succes beschouwen?

%---------- Stand van zaken ---------------------------------------------------

\section{Literatuurstudie}%
\label{sec:literatuurstudie}
% Voor literatuurverwijzingen zijn er twee belangrijke commando's:
% \autocite{KEY} => (Auteur, jaartal) Gebruik dit als de naam van de auteur
%   geen onderdeel is van de zin.
% \textcite{KEY} => Auteur (jaartal)  Gebruik dit als de auteursnaam wel een
%   functie heeft in de zin (bv. ``Uit onderzoek door Doll & Hill (1954) bleek
%   ...'')

\subsection{Kleurenblindheid op Android}
Kleurenblindheid, waaronder protanopie, deuteranopie en tritanopie, ontstaat door defecten in de 
fotoreceptoren in het netvlies, waardoor het moeilijk wordt om kleuren correct waar te nemen \autocite{Salih2020}. 
Dit heeft invloed op de gebruikerservaring van mensen met deze aandoeningen, vooral in digitale omgevingen. 
Android biedt al verschillende apps die gebruikers helpen bij het herkennen van kleuren, bijvoorbeeld 
in wetenschappelijke toepassingen zoals titraties met fenolftaleïne. Volgens \textcite{Bandyopadhyay2017} verbeteren 
de apps niet alleen de precisie in laboratoriumresultaten, maar vergroten ze ook de algehele gebruikerservaring 
door visuele beperkingen te beperken. Daarnaast zijn er algoritmes zoals LMS Daltonization en LAB-correctie 
ontwikkeld, die de waarneming van kleuren verbeteren, wat vooral nuttig is in digitale en interactieve 
omgevingen \autocite{Baswaraju2020}. Dit toont aan hoe technologische innovaties de toegankelijkheid kunnen vergroten. 
Volgens \textcite{Crawford2024} is het naleven van de WCAG 2.1-normen essentieel voor het waarborgen van de 
toegankelijkheid van apps, zeker met de verplichtingen die zijn vastgelegd in de Europese Toegankelijkheidswet.
De implementatie van deze richtlijnen helpt ontwikkelaars bij het ontwerpen van apps die breder toegankelijk
zijn voor mensen met visuele beperkingen.
\subsection{Android SDK}
De Android Software Development Kit (SDK) biedt ontwikkelaars de nodige tools voor het bouwen 
van apps die niet alleen interactief en responsief zijn, maar ook rekening houden met 
de toegankelijkheid van gebruikers. Dit is van groot belang voor mensen met visuele beperkingen 
zoals kleurenblindheid. De SDK bevat verschillende componenten die essentieel zijn voor het 
ontwikkelen van apps die visueel aangepast kunnen worden aan de behoeften van de gebruiker, 
zoals UI-elementen, content providers en API-integraties. Volgens \textcite{Geeks2024} kunnen ontwikkelaars 
met de Android SDK gebruik maken van functies zoals TalkBack, een schermlezer die tekst hardop 
voorleest en het gemakkelijker maakt om door een app te navigeren voor visueel gehandicapte gebruikers. 
Dit vergroot de bruikbaarheid van apps, vooral voor mensen met kleurenblindheid. De juiste integratie 
van API's en interactieve elementen zijn essentieel voor het creëren van apps die toegankelijk zijn 
voor mensen met verschillende beperkingen. Door deze integratie kan de app zich aanpassen aan de 
visuele behoeften van de gebruiker, wat bijdraagt aan een inclusievere gebruikerservaring \autocite{Lee2011}.
\subsection{Android Toegankelijkheid}




%---------- Methodologie ------------------------------------------------------
\section{Methodologie}%
\label{sec:methodologie}

Hier beschrijf je hoe je van plan bent het onderzoek te voeren. Welke onderzoekstechniek ga je toepassen om elk van je onderzoeksvragen te beantwoorden? Gebruik je hiervoor literatuurstudie, interviews met belanghebbenden (bv.~voor requirements-analyse), experimenten, simulaties, vergelijkende studie, risico-analyse, PoC, \ldots?

Valt je onderwerp onder één van de typische soorten bachelorproeven die besproken zijn in de lessen Research Methods (bv.\ vergelijkende studie of risico-analyse)? Zorg er dan ook voor dat we duidelijk de verschillende stappen terug vinden die we verwachten in dit soort onderzoek!

Vermijd onderzoekstechnieken die geen objectieve, meetbare resultaten kunnen opleveren. Enquêtes, bijvoorbeeld, zijn voor een bachelorproef informatica meestal \textbf{niet geschikt}. De antwoorden zijn eerder meningen dan feiten en in de praktijk blijkt het ook bijzonder moeilijk om voldoende respondenten te vinden. Studenten die een enquête willen voeren, hebben meestal ook geen goede definitie van de populatie, waardoor ook niet kan aangetoond worden dat eventuele resultaten representatief zijn.

Uit dit onderdeel moet duidelijk naar voor komen dat je bachelorproef ook technisch voldoen\-de diepgang zal bevatten. Het zou niet kloppen als een bachelorproef informatica ook door bv.\ een student marketing zou kunnen uitgevoerd worden.

Je beschrijft ook al welke tools (hardware, software, diensten, \ldots) je denkt hiervoor te gebruiken of te ontwikkelen.

Probeer ook een tijdschatting te maken. Hoe lang zal je met elke fase van je onderzoek bezig zijn en wat zijn de concrete \emph{deliverables} in elke fase?

%---------- Verwachte resultaten ----------------------------------------------
\section{Verwacht resultaat, conclusie}%
\label{sec:verwachte_resultaten}

Hier beschrijf je welke resultaten je verwacht. Als je metingen en simulaties uitvoert, kan je hier al mock-ups maken van de grafieken samen met de verwachte conclusies. Benoem zeker al je assen en de onderdelen van de grafiek die je gaat gebruiken. Dit zorgt ervoor dat je concreet weet welk soort data je moet verzamelen en hoe je die moet meten.

Wat heeft de doelgroep van je onderzoek aan het resultaat? Op welke manier zorgt jouw bachelorproef voor een meerwaarde?

Hier beschrijf je wat je verwacht uit je onderzoek, met de motivatie waarom. Het is \textbf{niet} erg indien uit je onderzoek andere resultaten en conclusies vloeien dan dat je hier beschrijft: het is dan juist interessant om te onderzoeken waarom jouw hypothesen niet overeenkomen met de resultaten.

